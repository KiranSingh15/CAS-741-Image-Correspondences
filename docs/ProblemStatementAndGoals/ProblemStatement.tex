\documentclass{article}

\usepackage{tabularx}
\usepackage{booktabs}

\title{Problem Statement and Goals\\\progname}
\author{\authname}
\date{April 17, 2025}

%% Comments

\usepackage{color}

\newif\ifcomments\commentstrue %displays comments
%\newif\ifcomments\commentsfalse %so that comments do not display

\ifcomments
\newcommand{\authornote}[3]{\textcolor{#1}{[#3 ---#2]}}
\newcommand{\todo}[1]{\textcolor{red}{[TODO: #1]}}
\else
\newcommand{\authornote}[3]{}
\newcommand{\todo}[1]{}
\fi

\newcommand{\wss}[1]{\authornote{blue}{SS}{#1}} 
\newcommand{\plt}[1]{\authornote{magenta}{TPLT}{#1}} %For explanation of the template
\newcommand{\an}[1]{\authornote{cyan}{Author}{#1}}

%% Common Parts

\newcommand{\progname}{ProgName} % PUT YOUR PROGRAM NAME HERE
\newcommand{\authname}{Team \#, Team Name
\\ Student 1 name
\\ Student 2 name
\\ Student 3 name
\\ Student 4 name} % AUTHOR NAMES                  

\usepackage{hyperref}
    \hypersetup{colorlinks=true, linkcolor=blue, citecolor=blue, filecolor=blue,
                urlcolor=blue, unicode=false}
    \urlstyle{same}
                                


\begin{document}

\maketitle

\begin{table}[hp]
\caption{Revision History} \label{TblRevisionHistory}
\begin{tabularx}{\textwidth}{llX}
\toprule
\textbf{Date} & \textbf{Developer(s)} & \textbf{Change}\\
\midrule
2025-01-17 & Kiran Singh & Initial Release\\
2025-04-17 & Kiran Singh & Final Release\\

\bottomrule
\end{tabularx}
\end{table}

\section{Problem Statement}
\subsection{Overview}
Camera calibration procedures are essential to ensure that cameras can collect 
useful data for robotic systems. Current calibration methods face trade-offs 
such as reliance on local solvers, synchronized imagery, and predefined targets 
within the environment. Current research in the Autonomous Robotics and Convex 
Optimization (ARCO) Lab work to design new calibration algorithms to perform 
multi-camera calibration that overcomes the specified limitations. This work 
will be achieved in two parts.

\begin{enumerate}
    \item A front-end system to interface with camera data such that:

    \begin{itemize}
        \item identifies correspondences between any given image frame between 
        camera data.

        \item computes the corresponding spatial transforms 
        between coordinate frames for cameras and correspondences.
    \end{itemize}

    \item A back-end system that performs 3D point-cloud registration and 
    outlier removal.
\end{enumerate}  


\subsection{Problem}
To satisfy administrative and schedule constraints of the Winter 2025 development schedule, the scope of the 
course project will address the development of the front-end software that identifies 
correspondences amongst camera imagery across selected poses.

\subsection{Inputs and Outputs}
\subsubsection{Inputs}
\begin{enumerate}
    \item The stream of imagery for each camera, taken at any given pose.
    \item the methods used to process the imagery.
    \item the specific parameters used to tune the system response for each image processing method.
\end{enumerate}  

\subsubsection{Outputs}
\begin{enumerate}
    \item A flag of whether correspondences have been identified.
    \item A list that outlines the graph nodes produced for any given camera 
    with another camera.
    \item An list of all identified correspondences.
    \item An flag for each transform that needs to be calculated as part 
    of downstream operations.
\end{enumerate}  

\subsection{Stakeholders}
Stakeholders will primarily be composed of roboticists that need to perform some 
form of extrinsic camera calibration. 
Though the majority of stakeholders are expected to be academic roboticists, 
industrial roboticists may wish to used this software as a 
commercial-off-the-shelf option to perform camera calibration for mobile 
robots, robotic manipulators, and aerial platforms.

\subsection{Environment}
The intent of this software is to be versatile in its deployment. Therefore, 
it should be compatible with standard operations such as Windows 10 or higher, 
MacOS, and Linux OS.
\newline
\newline
The software algorithm is expected to be developed in non-real time prior to 
upload to an embedded system. Therefore, as the algorithm itself will not be 
computed in real-time, development of the algorithm itself should not be 
constrained by additional memory limitations that are common for embedded 
systems.

\section{Goals}
This project will produce scientific computing software that facilitates 
robust extrinsic camera calibration for robotic systems without overlapping 
fields of view between its cameras. The primary goals are outlined below.

\begin{enumerate}
    \item Given a stream of imagery data from a set of cameras with a 
    corresponding robot pose, P, compare each image 
    between cameras to identify overlap in features as correspondences.

    \item Formulate a graph that, for each image from a given camera, defines 
    the identified correspondences with other imagery from other cameras as 
    edges, or links.

    \item Using the generated graph, define the transforms to be calculated 
    using the identified image correspondences. 
\end{enumerate}

\section{Stretch Goals}
\begin{enumerate}
    \item The software should define a quality metric that estimates the 
    portion of outliers amongst the identified points and correspondences. 
    This metric can be compared with the performance of outlier detection 
    and removal as part downstream operations.
\end{enumerate}

\section{Challenge Level and Extras}
This project is designed as a graduate research project with a corresponding 
degree of difficulty. The method of how correspondences are defined has been 
left to be ambiguous at this stage and may constitute that this software be 
defined as a family of products.
\newline
\newline
% deleted the submission of a user manual per the update to the Repos.CSV file.

\end{document}
