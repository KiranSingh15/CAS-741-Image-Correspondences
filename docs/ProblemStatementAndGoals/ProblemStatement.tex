\documentclass{article}

\usepackage{tabularx}
\usepackage{booktabs}

\title{Problem Statement and Goals\\\progname}

\author{\authname}

\date{January 17, 2025}

%% Comments

\usepackage{color}

\newif\ifcomments\commentstrue %displays comments
%\newif\ifcomments\commentsfalse %so that comments do not display

\ifcomments
\newcommand{\authornote}[3]{\textcolor{#1}{[#3 ---#2]}}
\newcommand{\todo}[1]{\textcolor{red}{[TODO: #1]}}
\else
\newcommand{\authornote}[3]{}
\newcommand{\todo}[1]{}
\fi

\newcommand{\wss}[1]{\authornote{blue}{SS}{#1}} 
\newcommand{\plt}[1]{\authornote{magenta}{TPLT}{#1}} %For explanation of the template
\newcommand{\an}[1]{\authornote{cyan}{Author}{#1}}

%% Common Parts

\newcommand{\progname}{ProgName} % PUT YOUR PROGRAM NAME HERE
\newcommand{\authname}{Team \#, Team Name
\\ Student 1 name
\\ Student 2 name
\\ Student 3 name
\\ Student 4 name} % AUTHOR NAMES                  

\usepackage{hyperref}
    \hypersetup{colorlinks=true, linkcolor=blue, citecolor=blue, filecolor=blue,
                urlcolor=blue, unicode=false}
    \urlstyle{same}
                                


\begin{document}

\maketitle

\begin{table}[hp]
\caption{Revision History} \label{TblRevisionHistory}
\begin{tabularx}{\textwidth}{llX}
\toprule
\textbf{Date} & \textbf{Developer(s)} & \textbf{Change}\\
\midrule
17.01.2025 & Kiran Singh & Initial Release\\
\bottomrule
\end{tabularx}
\end{table}

\section{Problem Statement}

\wss{You should check your problem statement with the
\href{https://github.com/smiths/capTemplate/blob/main/docs/Checklists/ProbState-Checklist.pdf}
{problem statement checklist}.} 

\wss{You can change the section headings, as long as you include the required
information.}

\subsection{Problem}

\subsection{Inputs and Outputs}
\subsubsection{Inputs}
\begin{enumerate}
    \item The quantity of active cameras.
    \item The intrinsic calibration parameters for each camera.
    \item The stream of imagery for each camera, taken at any given pose.
\end{enumerate}  

\subsubsection{Outputs}
\begin{enumerate}
    \item A flag of whether correspondences have been identified.
    \item A list that outlines the graph nodes produced for any given camera with another camera.
    \item An outline of all identified correspondences.
    \item An flag to indicate whether transforms need to be calculated as part of downstream operations.
\end{enumerate}  

\subsection{Stakeholders}
Stakeholders will primarily be composed of roboticists that need to perform some form of extrinsic camera calibration. 
Though the majority of stakeholders are expected to be academic roboticists, industrial roboticists may wish to used this software
as a commercial-off-the-shelf option to perform camera calibration for mobile robots, robotic manipulators, and aerial platforms.

\subsection{Environment}
The intent of this software is to be versatile in its deployment. Therefore, it should be compatible with standard operations 
such as Windows 10 or higher, MacOS, and Linux OS.

The software algorithm is expected to be developed in non-real time prior to upload to an embedded system. 
Therefore, as the algorithm itself will not be computed in real-time, development of the algorithm itself should not 
be constrained by additional memory limitations that are common for embedded systems.

\section{Goals}
This project will produce scientific computing software that facilitates robust extrinsic camera calibration for robotic systems without overlapping fields of view between its cameras. The primary goals are outlined below.
\begin{enumerate}
    \item Accept non-synchronous imagery from multiple imagery streams.
    \item Compare images from each imagery stream to identify correspondences between each image.
    \item Formulate a graph between each image with identified correspondences.
    \item Identify what camera transforms need to be calculated as part of downstream operations using the defined graph network.
\end{enumerate}

\section{Stretch Goals}
Outline a quality metric, such as confidence, that may be used to estimate what correspondences and points between images may be outliers. 
This metric can be compared with the performance of outlier detection and removal as part downstream operations.

\section{Challenge Level and Extras}
This project is designed as a graduate research project with a corresponding degree of difficulty. The method of how correspondences are defined has been left to be ambiguous at this stage and may constitute that this software be defined as a family of products.
\\A user manual will be provided as an 'extra' submission for this project.

\newpage{}

\section*{Appendix --- Reflection}

\wss{Not required for CAS 741}

The purpose of reflection questions is to give you a chance to assess your own
learning and that of your group as a whole, and to find ways to improve in the
future. Reflection is an important part of the learning process.  Reflection is
also an essential component of a successful software development process.  

Reflections are most interesting and useful when they're honest, even if the
stories they tell are imperfect. You will be marked based on your depth of
thought and analysis, and not based on the content of the reflections
themselves. Thus, for full marks we encourage you to answer openly and honestly
and to avoid simply writing ``what you think the evaluator wants to hear.''

Please answer the following questions.  Some questions can be answered on the
team level, but where appropriate, each team member should write their own
response:


\begin{enumerate}
    \item What went well while writing this deliverable? 
    \item What pain points did you experience during this deliverable, and how
    did you resolve them?
    \item How did you and your team adjust the scope of your goals to ensure
    they are suitable for a Capstone project (not overly ambitious but also of
    appropriate complexity for a senior design project)?
\end{enumerate}  

\end{document}